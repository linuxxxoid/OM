\section{Выводы}
Подводя итог, я узнала новые алгоритмы безусловной оптимизации, выполнив лабораторную работу по курсу \enquote{Методы оптимизации}. Эти методы позволяют находить минимумы функций в процессе иттерационных приближений. Также я выявила для себя их преимущества и недостатки. \\

Все изученные алгоритмы показались мне довольно интересными. Выделился алгоритм  Нелдера-Мида, так как он не требовал вычисления градиента и матрицы Гёссе. Вследствие этого, появляется возможность работать с негладкими функциями и к тому же исчезают ограничения на положительную определенность матрицы.\\

Это было интересным опытом, надеюсь, что мне предоставится возможность в будущем применить свои знания, полученные при выполнении данной лабораторной.

\pagebreak
